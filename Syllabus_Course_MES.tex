Curso-Taller

Tópicos de Evaluación de Estrategias de Manejo (EEM) (8 horas lectivas)

Profesor.  Cristian M. Canales - PUCV

Dia 1: 23 de junio 11.00 – 13.00 hrs
- Introducción a la Evaluación de Estrategias/Procedimiento de Manejo pesquero (EEM)
- Componentes de la EEM
- Definición y tipos de reglas de control/decisión de capturas (RCC): basadas en la evaluación de stock vs basadas en indicadores empíricos
- Identificación de objetivos/variables operacionales/desempeño
- Ejemplos

Dia 2: 30 de junio 11.00 – 13.00 hrs
- Ecuaciones de dinámica poblacional y diseño de Modelos Operativos (MO)
- Algunas distribuciones de probabilidad y aleatorización (R-studio). Simulación
- Identificación de fuentes de incertidumbre
- Implementación de un MO (R-studio). Modelamiento de procesos biológicos, pesqueros y decisionales (RCC)
- Condicionamiento e identificación de MO alternativos.

Dia 3: 7 de julio 11.00 – 13.00 hrs
- Simulación de un procedimiento/estrategia de manejo (R-studio)
- Compilación de resultados, variables y métricas de desempeño

Dia 4: 14 de julio 11.00 – 13.00 hrs
- Generación de resultados útiles para la toma de decisiones/acuerdos
- Diagramas de radar, box-plot y tablas de decisión
